%% This style is provided for the ICSE 2015 main conference,
%% ICSE 2015 co-located events, and ICSE 2015 workshops.

%% bare_conf_ICSE15.tex
%% V1.4
%% 2014/05/22


%% This is a skeleton file demonstrating the use of IEEEtran.cls
%% (requires IEEEtran.cls version 1.7 or later) with an IEEE conference paper.
%%
%% Support sites:
%% http://www.michaelshell.org/tex/ieeetran/
%% http://www.ctan.org/tex-archive/macros/latex/contrib/IEEEtran/
%% and
%% http://www.ieee.org/

%%*************************************************************************
%% Legal Notice:
%% This code is offered as-is without any warranty either expressed or
%% implied; without even the implied warranty of MERCHANTABILITY or
%% FITNESS FOR A PARTICULAR PURPOSE!
%% User assumes all risk.
%% In no event shall IEEE or any contributor to this code be liable for
%% any damages or losses, including, but not limited to, incidental,
%% consequential, or any other damages, resulting from the use or misuse
%% of any information contained here.
%%
%% All comments are the opinions of their respective authors and are not
%% necessarily endorsed by the IEEE.
%%
%% This work is distributed under the LaTeX Project Public License (LPPL)
%% ( http://www.latex-project.org/ ) version 1.3, and may be freely used,
%% distributed and modified. A copy of the LPPL, version 1.3, is included
%% in the base LaTeX documentation of all distributions of LaTeX released
%% 2003/12/01 or later.
%% Retain all contribution notices and credits.
%% ** Modified files should be clearly indicated as such, including  **
%% ** renaming them and changing author support contact information. **
%%
%% File list of work: IEEEtran.cls, IEEEtran_HOWTO.pdf, bare_adv.tex,
%%                    bare_conf.tex, bare_jrnl.tex, bare_jrnl_compsoc.tex
%%*************************************************************************

% *** Authors should verify (and, if needed, correct) their LaTeX system  ***
% *** with the testflow diagnostic prior to trusting their LaTeX platform ***
% *** with production work. IEEE's font choices can trigger bugs that do  ***
% *** not appear when using other class files.                            ***
% The testflow support page is at:
% http://www.michaelshell.org/tex/testflow/



% Note that the a4paper option is mainly intended so that authors in
% countries using A4 can easily print to A4 and see how their papers will
% look in print - the typesetting of the document will not typically be
% affected with changes in paper size (but the bottom and side margins will).
% Use the testflow package mentioned above to verify correct handling of
% both paper sizes by the user's LaTeX system.
%
% Also note that the "draftcls" or "draftclsnofoot", not "draft", option
% should be used if it is desired that the figures are to be displayed in
% draft mode.
%
\documentclass[conference]{sig-alternate}

%
% If IEEEtran.cls has not been installed into the LaTeX system files,
% manually specify the path to it like:
% \documentclass[conference]{../sty/IEEEtran}



% Some very useful LaTeX packages include:
% (uncomment the ones you want to load)

%%%%%%%%%%%%%%%%%%%%%%%%%%%%%%%%%%%%%%%%
% Put edit comments in a really ugly standout display
\usepackage{ifthen}
\usepackage{amssymb}
\usepackage{arydshln}
\usepackage{hhline}
\usepackage{xspace}
\usepackage{hyperref}
\usepackage{xcolor}
\usepackage{graphicx}
\usepackage{fixltx2e}
\usepackage{pdfpages}
\usepackage{caption}
\usepackage{subcaption}
\usepackage{balance}

%%%%%%%%%%%%%%%%%%%%%%%%%%%%%%%%%%%%%%%%%%%%%%%%%%
% Markup macros for proof-reading
\usepackage[normalem]{ulem} % for \sout
\usepackage{xcolor}
\newcommand{\ra}{$\rightarrow$}
\newcommand{\ugh}[1]{\textcolor{red}{\uwave{#1}}} % please rephrase
\newcommand{\ins}[1]{\textcolor{blue}{\uline{#1}}} % please insert
\newcommand{\del}[1]{\textcolor{red}{\sout{#1}}} % please delete
\newcommand{\chg}[2]{\textcolor{red}{\sout{#1}}{\ra}\textcolor{blue}{\uline{#2}}} % please change
%%%%%%%%%%%%%%%%%%%%%%%%%%%%%%%%%%%%%%%%
% Put edit comments in a really ugly standout display
\usepackage{ifthen}
\usepackage{amssymb}
\newboolean{showcomments}
\setboolean{showcomments}{false} % toggle to show or hide comments
\ifthenelse{\boolean{showcomments}}
  {\newcommand{\Note}[3]{\fcolorbox{gray}{#3}{\bfseries\sffamily\scriptsize#1}{\sf\small$\blacktriangleright$\textit{#2}$\blacktriangleleft$}}}
  {\newcommand{\Note}[3]{}}
\newcommand{\todo}[1]{\Note{todo}{#1}{yellow}}
%%%%%%%%%%%%%%%%%%%%%%%%%%%%%%%%%%%%%%%%
\newcommand{\cat}[1]{\Note{C. Albert}{#1}{pink}}
\newcommand{\gm}[1]{\Note{Gail}{#1}{green}}
%%%%%%%%%%%%%%%%%%%%%%%%%%%%%%%%%%%%%%%%
\newcommand\transitive{transition\xspace}
\newcommand\issue{task\xspace}
\newcommand\issues{tasks\xspace}
\newcommand\tasks{task\xspace}
\newcommand\task{tasks\xspace}
\newcommand\mylyn{\texttt{Mylyn}\xspace}
\newcommand\connect{\texttt{Connect}\xspace}
\newcommand\hbase{\texttt{HBase}\xspace}
\newcommand\apache{\texttt{Apache}\xspace}
\newcommand\firefox{\texttt{Firefox}\xspace}
\newcommand\eclipse{\texttt{Eclipse}\xspace}
%%%%%%%%%%%%%%%%%%%%%%%%%%%%%%%%%%%%%%%%
%Spacing changes!
%\vspace{-0.2em}



% *** Do not adjust lengths that control margins, column widths, etc. ***
% *** Do not use packages that alter fonts (such as pslatex).         ***
% There should be no need to do such things with IEEEtran.cls V1.6 and later.
% (Unless specifically asked to do so by the jmynal or conference you plan
% to submit to, of cmyse. )


% correct bad hyphenation here
\hyphenation{reposir-ories}


\begin{document}
%
% paper title
% can use linebreaks \\ within to get better formatting as desired
\title{A Study of Relationships in Issue Repositories}

%\numberofauthors{4} %  in this sample file, there are a *total*
%% of EIGHT authors. SIX appear on the 'first-page' (for formatting
%% reasons) and the remaining two appear in the \additionalauthors section.
%%
%\author{
%

%}

\numberofauthors{4} %  in this sample file, there are a *total*
% of EIGHT authors. SIX appear on the 'first-page' (for formatting
% reasons) and the remaining two appear in the \additionalauthors section.
%
%\author{
%% You can go ahead and credit any number of authors here,
%% e.g. one 'row of three' or two rows (consisting of one row of three
%% and a second row of one, two or three).
%%
%% The command \alignauthor (no curly braces needed) should
%% precede each author name, affiliation/snail-mail address and
%% e-mail address. Additionally, tag each line of
%% affiliation/address with \affaddr, and tag the
%% e-mail address with \email.
%%
%% 1st. author
%\alignauthor C. Albert Thompson \\
%%       \affaddr{Department of Computer Science}\\
%       \affaddr{Univ. of British Columbia}\\
%%       \affaddr{Vancouver, B.C. Canada}\\
%       \email{leetcat@cs.ubc.ca}
%% 2nd. author
%\alignauthor Gail C. Murphy\\
%%%        \affaddr{Department of Computer Science}\\
%        \affaddr{Univ. of British Columbia}\\
%%%        \affaddr{Vancouver, B.C. Canada}\\
%        \email{murphy@cs.ubc.ca}
%% 3rd. author
%\and
% \alignauthor Marc Palyart\\
%%%        \affaddr{Department of Computer Science}\\
%        \affaddr{Univ. of British Columbia}\\
%%%        \affaddr{Vancouver, B.C. Canada}\\
%        \email{mpalyart@cs.ubc.ca}
%%%          % use '\and' if you need 'another row' of author names
%% 4th. author
%\alignauthor Marko Gasparic\\
%       \affaddr{Free Univ. of Bozen-Bolzano}\\
%%%       \affaddr{Bozen-Bolzano, Italy}\\
%       \email{marko.gasparic@stud-inf.unibz.it}
%}
% make the title area

\author{
	\begin{tabular}{@{}p{.5in}@{}p{.5in}@{}p{.5in}@{}p{.5in}@{}p{.5in}@{}p{.5in}@{}p{.5in}@{}p{.5in}@{}p{.5in}@{}p{.5in}@{}p{.5in}@{}p{.5in}@{}}
		\multicolumn{3}{p{1.7in}}{\centering C. Albert Thompson*} & \multicolumn{3}{p{1.5in}}{\centering Gail C. Murphy*} &
		\multicolumn{3}{p{1.5in}}{\centering Marc Palyart\textsuperscript*} & \multicolumn{3}{p{1.5in}}{\centering Marko Gasparic\textsuperscript{$+$}} \\
	\end{tabular}
	\\
	\begin{tabular}{@{}p{2.25in}@{}p{1.5in}@{}p{2.25in}@{}}
		\centerline{*\affaddr{University of British Columbia}}                           &   & \centerline{\textsuperscript{$+$}\affaddr{Free University of Bozen-Bolzano}} \\ \\
		\vspace{-17pt}\centerline{\affaddr{Vancouver, BC, Canada}}                       &   & \vspace{-17pt}\centerline{\affaddr{Bozen-Bolzano, Italy}}                    \\
		\vspace{-17pt}\centerline{{\normalsize \{leetcat, murphy, mpalyart\}@cs.ubc.ca}} &   & \vspace{-17pt}\centerline{{\normalsize marko.gasparic@stud-inf.unibz.it}}
	\end{tabular}
}




\maketitle


\begin{abstract}
	Software developers use issue trackers and define issues as a means of
	specifying work to be performed on a system and defects that require
	fixing.  Despite the manual effort developers invest in specifying
	many relationships between issues, the vast majority of techniques
	aimed at improving software development that rely on issues ignore
	relationship information.  In this paper, we investigate how
	developers use relationships between issues, finding that many
	relationships are often specified and that the most prevalent use of
	relationships is to describe how work should be broken into smaller
	pieces.  We also found that there are few patterns in how words are
	used between parent and child issues involved in a work breakdown
	relationship, and through manual coding, determined seven codes that
	describe the semantic meanings of the work breakdown relationships
	specified across three open source systems.  We describe how the
	recognition of work breakdown relationships may be used to improve
	existing software development techniques and how the automatic
	identification of work breakdown relationships opens up new
	possibilities for auto-generation of child issues and consistency
	checkers for work to be performed on a system.



	%% Software developers use issue repositories for many purposes, from
	%% recording features and how the features might be implemented to
	%% defects that have been found and how those defects might be fixed.
	%% The ways in which developers go about their work are recorded in issue
	%% repositories in part by the organization of issue into hierarchies.
	%% To better understand the issue hierarchies used in issue repositories,
	%% we undertook a study of the hierarchies in the repositories for two
	%% open source systems.  We applied two automated methods to analyze
	%% syntactic patterns in relationships between issues. We then qualitatively
	%% analyzed a subset of these pairs for their semantic meaning,
	%% developing seven tags to represent this meaning, we
	%% found that most issues in the three systems have child issues
	%% which reference work breakdown.
	%% This study of relationships in the of the structure of
	%% issues in software development may aid in the development of
	%% tools to recommend work breakdown that a developer might need to undertake,
	%% in determining where processes are not being completely followed, or
	%% in enhancing existing techniques for automatically determining who
	%% should undertake particular tasks or which issues are duplicates of
	%% each other.
\end{abstract}

% no keywords


%\category{D.2.6}{Software Engineering}{Programming Environments}
%\category{K.6.3}{Software Management}{Software development}
%
%\IEEEterms{Algorithms}

%\begin{keywords}
%repository mining, task categorization
%\end{keywords}


% For peer review papers, you can put extra information on the cover
% page as needed:
% \ifCLASSOPTIONpeerreview
% \begin{center} \bfseries EDICS Category: 3-BBND \end{center}
% \fi
%
% For peerreview papers, this IEEEtran command inserts a page break and                                                                       
% creates the second title. It will be ignored for other modes.



% You must have at least 2 lines in the paragraph with the drop letter
% (should never be an issue)
\input{sections/intro}

\input{sections/related}

\input{sections/relationship}

\input{sections/patterns}

\input{sections/semantic}

%\input{sections/hierarchy}

\input{sections/discussion}

\input{sections/summary}

%\input{sections/questions}

\input{sections/acknowledgments}

% trigger a \newpage just before the given reference
% number - used to balance the columns on the last page
% adjust value as needed - may need to be readjusted if
% the document is modified later
%\IEEEtriggeratref{8}
% The "triggered" command can be changed if desired:
%\IEEEtriggercmd{\enlargethispage{-5in}}
\balance

\bibliographystyle{abbrv}
\bibliography{ICSE2016_Categorization}

% that's all folks
\end{document}
